\documentclass[]{article}
\usepackage{amssymb,amsmath}
\usepackage{ifxetex,ifluatex}
\ifxetex
  \usepackage{fontspec,xltxtra,xunicode}
  \defaultfontfeatures{Mapping=tex-text,Scale=MatchLowercase}
\else
  \ifluatex
    \usepackage{fontspec}
    \defaultfontfeatures{Mapping=tex-text,Scale=MatchLowercase}
  \else
    \usepackage[utf8]{inputenc}
  \fi
\fi
\usepackage{color}
\usepackage{fancyvrb}
\DefineShortVerb[commandchars=\\\{\}]{\|}
\DefineVerbatimEnvironment{Highlighting}{Verbatim}{commandchars=\\\{\}}
% Add ',fontsize=\small' for more characters per line
\newenvironment{Shaded}{}{}
\newcommand{\KeywordTok}[1]{\textcolor[rgb]{0.00,0.44,0.13}{\textbf{{#1}}}}
\newcommand{\DataTypeTok}[1]{\textcolor[rgb]{0.56,0.13,0.00}{{#1}}}
\newcommand{\DecValTok}[1]{\textcolor[rgb]{0.25,0.63,0.44}{{#1}}}
\newcommand{\BaseNTok}[1]{\textcolor[rgb]{0.25,0.63,0.44}{{#1}}}
\newcommand{\FloatTok}[1]{\textcolor[rgb]{0.25,0.63,0.44}{{#1}}}
\newcommand{\CharTok}[1]{\textcolor[rgb]{0.25,0.44,0.63}{{#1}}}
\newcommand{\StringTok}[1]{\textcolor[rgb]{0.25,0.44,0.63}{{#1}}}
\newcommand{\CommentTok}[1]{\textcolor[rgb]{0.38,0.63,0.69}{\textit{{#1}}}}
\newcommand{\OtherTok}[1]{\textcolor[rgb]{0.00,0.44,0.13}{{#1}}}
\newcommand{\AlertTok}[1]{\textcolor[rgb]{1.00,0.00,0.00}{\textbf{{#1}}}}
\newcommand{\FunctionTok}[1]{\textcolor[rgb]{0.02,0.16,0.49}{{#1}}}
\newcommand{\RegionMarkerTok}[1]{{#1}}
\newcommand{\ErrorTok}[1]{\textcolor[rgb]{1.00,0.00,0.00}{\textbf{{#1}}}}
\newcommand{\NormalTok}[1]{{#1}}
\ifxetex
  \usepackage[setpagesize=false, % page size defined by xetex
              unicode=false, % unicode breaks when used with xetex
              xetex,
              colorlinks=true,
              linkcolor=blue]{hyperref}
\else
  \usepackage[unicode=true,
              colorlinks=true,
              linkcolor=blue]{hyperref}
\fi
\hypersetup{breaklinks=true, pdfborder={0 0 0}}
\setlength{\parindent}{0pt}
\setlength{\parskip}{6pt plus 2pt minus 1pt}
\setlength{\emergencystretch}{3em}  % prevent overfull lines
\setcounter{secnumdepth}{0}


\begin{document}

I am finding myself more and more drawn to markdown rather then tex/Rnw
as my standard format (not least of which is the ease of displaying the
files on github, particularly now that we have automatic image
uploading). One thing I miss from latex is the citation commands. (I
understand these can be provided to markdown via Pandoc, but I'd like to
simply have to knit the document, and not then run it through pandoc,
latex, or another interpreter). I've taken a little whack at generating
in-text citations using knitr and other R tools.

\subsubsection{DOI Approach}

I've put some simple functions in a \texttt{knitcitations} package. The
functions use the crossref API to grab citation information given a doi,
so I don't have to generate a bibtex file for papers I'm reading,
(inspired by the
\href{http://wordpress.org/extend/plugins/kcite/}{kcite} package for
Wordpress). One can grab my package from github

\begin{Shaded}
\begin{Highlighting}[]
\KeywordTok{library}\NormalTok{(devtools)}
\KeywordTok{install_github}\NormalTok{(}\StringTok{"knitcitations"}\NormalTok{, }\StringTok{"cboettig"}\NormalTok{)}
\end{Highlighting}
\end{Shaded}
and load the package

\begin{Shaded}
\begin{Highlighting}[]
\KeywordTok{require}\NormalTok{(knitcitations)}
\end{Highlighting}
\end{Shaded}
Then we can generate a citation given a doi with the \texttt{ref}
function:

\subsubsection{Bibtex Approach}

If we have a bibtex file, we can use this for the citations as well.
Let's start off by getting ourselves a bibtex file from some of R's
packages:

\begin{Shaded}
\begin{Highlighting}[]
\KeywordTok{library}\NormalTok{(bibtex)}
\KeywordTok{write.bib}\NormalTok{(}\KeywordTok{c}\NormalTok{(}\StringTok{'bibtex'}\NormalTok{, }\StringTok{'knitr'}\NormalTok{, }\StringTok{'knitcitations'}\NormalTok{), }\DataTypeTok{file=}\StringTok{"example.bib"}\NormalTok{)}
\end{Highlighting}
\end{Shaded}
Now we can simply read in the bibtex files:

\begin{Shaded}
\begin{Highlighting}[]
\NormalTok{biblio <- }\KeywordTok{read.bib}\NormalTok{(}\StringTok{"example.bib"}\NormalTok{)}
\NormalTok{biblio[[}\DecValTok{1}\NormalTok{]]}
\end{Highlighting}
\end{Shaded}
\begin{verbatim}
Francois R (2011). _bibtex: bibtex parser_. R package version
0.3-1/r332, <URL: http://R-Forge.R-project.org/projects/highlight/>.
\end{verbatim}
(This would be much more awesome if we could generate keys on write.bib
and use those bibtex keys, instead of the index value,
\texttt{{[}{[}1{]}{]}}, to generate the citation.)

\subsubsection{Using the inline citations}

Now that we can get citation information from bibtex files or dois, we
need a way to insert these citations into the text. I've written a
simple \texttt{citep} print inline citations that would just use a given
shortened format (e.g.~author-year) and add the citation to a
\texttt{works\_cited} object, which we could then use to generate the
full citation information at the end. We can generate inline citations
by giving a doi, bibentry object, or a list thereof, into inline knitr
code block. Thus we can use the line
\texttt{citep("10.1111/j.1461-0248.2005.00827.x")} to generate a
parenthetical citation, (Halpern \emph{et. al.} 2006). We can also
generate textual citations with \texttt{citet}, such as Francois,
(2011). Parenthetical citations can take more than one entry, (Xie,
2012; Boettiger, 2012).

\subsubsection{Generating the final bibliography}

As we go along adding inline citations, R stores the list of citation
info. Then at the end of the document, use this command to print the
bibliography generated by the use of our inline citations.

\begin{Shaded}
\begin{Highlighting}[]
\KeywordTok{bibliography}\NormalTok{()}
\end{Highlighting}
\end{Shaded}
Halpern B, Regan H, Possingham H and McCarthy M (2006). ``Accounting for
uncertainty in marine reserve design.'' \emph{Ecology Letters},
\emph{9}. ISSN 1461-023X, .

Francois R (2011). \emph{bibtex: bibtex parser}. R package version
0.3-1/r332, .

Xie Y (2012). \emph{knitr: A general-purpose package for dynamic report
generation in R}. R package version 0.4.1, .

Boettiger C (2012). \emph{knitcitations: Citations for knitr markdown
files}. R package version 0.0-1.

I hope to add markup to format this a bit more nicely later. For
instance, we want the links to appear as real links. Additionally, we
may want to add markup around the citations, such as the reason for the
citation into the link using the
\href{http://speroni.web.cs.unibo.it/cgi-bin/lode/req.py?req=http:/purl.org/spar/cito}{Citation
Typing Ontology}.

\end{document}
